\chapter{Uvod}

%U skladu sa dobrom istraživačkom praksom, uvodno poglavlje rada prvog ciklusa studija bi trebao sadržavati bar sljedeće elemente:
%\begin{itemize}
%\item {obrazloženje teme,}
%\item {opis strukture rada.}
%\end{itemize}

%U narednom tekstu će detaljnije biti obrazložena svaka od tačaka.

\section{Obrazloženje teme}

Postupci minimizacije se mogu gledati kao jedno stablo. Na prvom stepenu možemo izabrati tip minimizacije i izabraćemo gradijentski. od svih mogućih gradijentskih, izabraćemo Nesterovljev ubrzani gradijentni metod.
Interesantan je, lići na nešto što bi radili na IF1. Ova tema je od izuzetnog značaja, jer nam ovisi ocjena o njoj, a neke od literatura koje smo koristili su nađene na internetu i knjigama.

%U ovoj sekciji autor je dužan da obrazloži koja tema ili problem će biti analizirani ili istraživani, te zbog čega je upravo ova tema pogodna i bitna za istraživanje. Pohvalno je napraviti pregled literature sa odgovarajućim referenciranjem na istu.

\section{Struktura rada}

Prvo ćemo dati teoretske osnove, zatim ćemo implementirati kod, vidjeti može li brže i vidjeti šta dalje.
%U ovoj sekciji je najbolje dati po jedan paragraf o svakom poglavlju iz rada. Potencirajte koji su glavni doprinosi svakog poglavlja, te kako su poglavlja medjusobno povezana.




