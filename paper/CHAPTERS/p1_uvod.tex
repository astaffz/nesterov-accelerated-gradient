\chapter{Uvod}

%U skladu sa dobrom istraživačkom praksom, uvodno poglavlje rada prvog ciklusa studija bi trebao sadržavati bar sljedeće elemente:
%\begin{itemize}
%\item {obrazloženje teme,}
%\item {opis strukture rada.}
%\end{itemize}

%U narednom tekstu će detaljnije biti obrazložena svaka od tačaka.

\section{Obrazloženje teme}

%Šta je minimizacija?

%Šta je gradijent?

%Šta je klasični momentum?

%Kako NAG pravi napredak u odnosu na klasični momentum?

%Koje ima NAG interpretacija?

%Kakva je brzina konvergencije?




Problem minimizacije funkcija predstavlja jedan od najvažnijih problema primijenjene matematike. U najopštijem obliku, cilj je odrediti tačku
\(\mathbf{x}  \in \mathbb{R} ^n\) u kojoj funkcija \(f(\mathbf{x})\) dostiže minimalnu vrijednost. Ovakvi problemi javljaju se u brojnim oblastima primijenjene matematike, fizike, ekonomije i savremenog mašinskog učenja.


Jedna od osnovnih metoda za rješavanje problema minimizacije diferencijabilnih funkcija jeste gradijentni metod. Gradijent funkcije predstavlja vektor parcijalnih izvoda i pokazuje pravac najvećeg rasta funkcije, pa se minimizacija postiže kretanjem u suprotnom smjeru gradijenta. Iako je metoda jednostavna i široko primjenjiva, njena konvergencija može biti spora, naročito kod loše uslovljenih problema.

Radi poboljšanja performansi razvijene su metode koje koriste dodatne informacije iz prethodnih iteracija. Jedan od prvih pristupa je metod momentuma, koji uvodi “inerciju” u proces optimizacije i na taj način ublažava oscilacije i ubrzava kretanje kroz ravnije dijelove funkcije.

Nesterov ubrzani gradijent metod predstavlja unapređenje klasičnog momentum pristupa. Za razliku od standardnog momentuma, kod Nesterovog metoda gradijent se računa u unaprijed predviđenoj tački, čime se dobija efikasniji i stabilniji postupak. Ova modifikacija dovodi do značajnog teorijskog poboljšanja brzine konvergencije.

Cilj ovog rada je da se prikaže teorijska osnova Nesterovog ubrzanog gradijentnog metoda, objasne njegove interpretacije, analiziraju osobine konvergencije, te implementira algoritam i ispita njegovo ponašanje na Rosenbrockovoj funkciji.


%Postupci minimizacije se mogu gledati kao jedno stablo. Na prvom stepenu možemo izabrati tip minimizacije i izabraćemo gradijentski. od svih mogućih gradijentskih, izabraćemo Nesterovljev ubrzani gradijentni metod.

%In a groundbreaking paper in 1983, Nesterov, Y. [Nes83] showed that a simple variant of gradient de- scent—called accelerated gradient descent and applicable to any 𝐿-smooth convex function—produces iterates with optimality gap 𝑓(𝑥𝑡) − 𝑓⋆ of order 1/𝑡2, as opposed to the 1/𝑡 rate seen in the previous lecture. The intuition behind accelerated gradient descent is notoriously hard to grasp. The original proof, rife with algebraic manipulations, is notoriously elusive and has led several authors to investi- gate what principles make acceleration possible at a deep level, hoping to generalize the fundamental principles beyond just gradient descent. These efforts include at least the following directions




% U ovoj sekciji autor je dužan da obrazloži koja tema ili problem će biti analizirani ili istraživani, te zbog čega je upravo ova tema pogodna i bitna za istraživanje. Pohvalno je napraviti pregled literature sa odgovarajućim referenciranjem na istu.

\section{Struktura rada}
%U ovoj sekciji je najbolje dati po jedan paragraf o svakom poglavlju iz rada. Potencirajte koji su glavni doprinosi svakog poglavlja, te kako su poglavlja medjusobno povezana.


Rad je organizovan u 4 poglavlja. U prvom poglavlju se opisuje struktura rada i teoretska osnova potrebna za razumjevanje Nesterovog ubrzanog gradijenta, kao i zašto i kada bi se trebao koristiti.

U drugom poglavlju će biti prikazane tri tipa implementacija Nesterovog ubrzanog gradijenta, to jest standardna varijanta, Sutskeverova modifikacija i Bengiova formulacija metode. Zatim će biti prikazan jedan praktičan primjer rada implementiranih funkcija i usporediće se njihovi rezultati.

Treće poglavlje će govoriti o primjenama algoritma u praksi, kao što su u dubokom učenju, modernim softverskim okvirima, optimizatorima te obradi signala i kompresovanog opažanja.


U četvrtom smo oformili zaključak.


\section{Teoretske osnove}

\subsection{Metod momentuma}

Jedno od prvih poboljšanja klasičnog gradijentnog metoda predstavlja metod momentuma, koji je uveo Boris Polyak. Osnovna ideja ovog pristupa jeste da se pri ažuriranju iteracija ne koristi samo trenutna vrijednost gradijenta, već i informacija o prethodnom kretanju algoritma.

\begin{equation}
    v_{t+1} = \mu v_t - \varepsilon \nabla  f(\theta _t)
\end{equation}

\begin{equation}
    \theta _{t+1} = \theta _t + v_{t+1}
\end{equation}

Gdje \(\varepsilon > 0 \) predstavlja parametar koraka, \(\mu \epsilon [0,1] \) parametar momentuma, a \( \nabla  f(\theta _t) \) gradijent.\cite{sutskever2013}

\subsection{Nesterov ubrzani gradijentni metod}

Metod Nesterovog ubrzaniog gradijenta je metod koji izgleda kao metod momentuma, ali nije u potpunosti isti.\cite{melville2016}

\begin{equation}
    v_{t+1} = \mu v_t - \varepsilon \nabla  f(\theta _t +\mu v_t)
\end{equation}

\begin{equation}
    \theta _{t+1} = \theta _t + v_{t+1}
\end{equation}

Razlika je u argumentu gradijenta \(+\mu v_t\), omogućuje metodu da brže popravi putanju kretanja od metod momentuma. \cite{sutskever2013}

% \subsection{Formalna analiza konvergencije}




