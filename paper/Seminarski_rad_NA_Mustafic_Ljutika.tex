% Ovo je glavni fajl za generiranje predloska dokumenta doktorske disertacije Elektrotehnickog fakulteta u Sarajevu. 
% Verzija: 11. maj 2018. godine 

% Autor: Emir Sokic , esokic@etf.unsa.ba
% Bazirano na predlosku koji se koristi na FER Zagreb

%%%%%%%%%%%%%%%%%%%%%%%%%%%%%%%%%%%%%%%%%%%%%%%%%%%%%%%%%%%%%%%%%%%%%%
%%%%%%%%%%%%%%%%%%%%%%%%% POSTAVKE %%%%%%%%%%%%%%%%%%%%%%%%%%%%%%%%%%%
%%%%%%%%%%%%%%%%%%%%%%%%% NE MIJENJATI %%%%%%%%%%%%%%%%%%%%%%%%%%%%%%%
%%%%%%%%%%%%%%%%%%%%%%%%%%%%%%%%%%%%%%%%%%%%%%%%%%%%%%%%%%%%%%%%%%%%%%

\documentclass[12pt,oneside, a4paper]{book}

%Ovo su paketi koji se koriste za kompajliranje dokumenta
%Vecina paketa je ukljucena po defaultu u texlive 2013 i novijim verzijama
\usepackage{silence}
\ErrorFilter{tocbasic}{Unknown message}
\usepackage{etex}%
\usepackage{xcolor}
\usepackage{graphicx}
\usepackage{rotating}
\usepackage{epsfig}
\usepackage{epstopdf}
\usepackage[T1]{fontenc}
\usepackage[utf8]{inputenc}%
\usepackage{cmap}%
\usepackage[croatian]{babel}
\usepackage[unicode]{hyperref}
\usepackage{mathptmx}
\usepackage{amscd}
\usepackage{amssymb}
\usepackage{amsmath}
\usepackage{amsfonts}
\usepackage[left=2.5cm,right=2.5cm,top=2.5cm,bottom=2.5cm]{geometry}
\setlength{\headheight}{14.5pt}
\usepackage{setspace} 
\usepackage{hhline}
\usepackage{enumerate}
\usepackage{delarray}
\usepackage{array}  
\usepackage{tabularx} 
\usepackage{multirow}  
\usepackage{wasysym}
\usepackage{subeqnarray}
\usepackage{pdflscape} % setting page into landscape view
\usepackage{enumitem} % for itemize lists
\usepackage[toc,page]{appendix}
\newcommand{\HRule}{\rule{\linewidth}{0.5mm}}
\usepackage{makeidx}
\usepackage{nomencl}
\usepackage[bf, font=small]{caption}
\usepackage[labelfont=small, font=small]{subcaption}
\usepackage{listings}
\lstset{basicstyle=\ttfamily,breaklines=true}
\usepackage{courier}
% Podesavanje izgleda zaglavlja i podnozja strana
\usepackage{fancyhdr} 
% Č, Ć, Đ, Ž, Š
\usepackage{newtxtext,newtxmath}

\fancypagestyle{plain}{%
  \fancyhf{}% Clear header/footer
  \fancyfoot[R]{{\thepage}}%
  \renewcommand{\headrulewidth}{0pt}%
}

% required for printing index
% use \index{name} in text
%\usepackage{makeidx}
%\makeindex
% required for printing nomenclature
% use \nomenclature{symbol}{description} in text
%\usepackage{nomencl}
%\makenomenclature
%\renewcommand{\nomname}{Popis oznaka}


%Opcionalno
%\linespread{1.3}
%\setlist{nolistsep}   % setting for itemize lists
%\renewcommand{\thefootnote}{\fnsymbol{footnote}}  % to get unnumbered footnotes

% Adding a dot after chapter number in TOC 
%\let\savenumberline\numberline
%\def\numberline#1{\savenumberline{#1.}}

%\pagestyle{fancyplain}

%\rfoot{\thepage}
% iskljucivanje broja strane iz Sadrzaja, Popisa slika i Popisa tabela
\AtBeginDocument{\addtocontents{toc}{\protect\thispagestyle{empty}}}
\AtBeginDocument{\addtocontents{lof}{\protect\thispagestyle{empty}}}
\AtBeginDocument{\addtocontents{lot}{\protect\thispagestyle{empty}}}

%\rhead{\slshape \nouppercase \leftmark}
%\lhead{} %delete left header


%Podesavanje izgleda referenci
\usepackage[square, numbers, sort]{natbib} 

%Promjena naziva pojedinih poglavlja sa Hrvatskog na Bosanski
% Bibliography u "Literatura"
\addto\captionscroatian{%
  \renewcommand{\bibname}{Literatura}
  \renewcommand{\tablename}{Tabela}
  \renewcommand{\nomname}{Popis oznaka}
  \renewcommand{\indexname}{Indeks pojmova}
  \renewcommand{\lstlistingname}{Program}
}
%"Popis tablica" u "Popis tabela"
\addto\captionscroatian{\renewcommand{\listtablename}{Popis tabela}}
\addto\captionscroatian{\renewcommand\appendixname{Prilog}}
\addto\captionscroatian{\renewcommand\appendixpagename{Prilozi}}
\renewcommand\appendixtocname{Prilozi}

\makeindex
\makenomenclature

%\usepackage{etoolbox}
%\patchcmd{\chapter}{\thispagestyle{plain}}{\thispagestyle{fancyplain}}{}{}


\begin{document}

%%%%%%%%%%%%%%%%%%%%%%%%%%%%%%%%%%%%%%%%%%%%%%%%%%%%%%%%%%%%%%%%%%%%%%
%%%%%%%%%%%%%%%%%%%%%%%%% OSNOVNI DOKUMENT %%%%%%%%%%%%%%%%%%%%%%%%%%%
%%%%%%%%%%%%%%%%%%%%%%%%%%%%%%%%%%%%%%%%%%%%%%%%%%%%%%%%%%%%%%%%%%%%%%

\frontmatter

%%%%%%%%%%%%%%%%%%%% NASLOVNA STRANA %%%%%%%%%%%%%%%%%%%%%%%%
\begin{titlepage}
  \begin{center}

    \includegraphics[width=0.25\textwidth]{Slike/etf-logo.png}~\\[0.1cm]
    \textsc{\Large Univerzitet u Sarajevu}\\[0.2cm]
    \textsc{\Large Elektrotehnički fakultet}\\[0.2cm]
    \textsc{\Large Odsjek za računarstvo i informatiku}\\[3cm]\HRule \\[0.5cm]
    {\huge \bfseries Nesterovljev ubrzani gradijentni metod za minimizaciju} \\[0.4cm]
    \HRule \\[0.5cm]

    \textsc{\Large Seminarski rad iz Numeričkih algoritama}\\[0.8cm]

    % Author and supervisor 
    \textbf{
      \Large Studenti:\\
      \Large Aid Mustafić\\
      \Large Zlatan Ljutika\\[1cm]
      \Large Profesor: \\[0.2cm]
      \Large Red. prof. dr Željko Jurić.}
    \vfill

    % Bottom of the page  
    {\large Sarajevo, januar 2026.}

  \end{center}
\end{titlepage}

% Sažetak (na Bosanskom jeziku) i Abstract (na Engleskom jeziku),
% postavka rada i izjava o autentičnoati su dati kao odvojeni fajlovi

%%%%%%%%%%%%%% SAŽETAK I ABSTRACT %%%%%%%%%%%%%%%%%%%%%%%%%%%%%%%%
%\pagebreak

\section*{Sažetak}


\begin{lstlisting}[language=Julia]
using Pkg;
Pkg.add("DifferentialEquations")
using DifferentialEquations
function opruga(du,u,p,t)
    du[1] = u[2];
    du[2] = -p[1]*u[1]/p[2]
end

p = [100.0;10.0]
u0 = [0.0;1.0]
tspan = (0.0,10.0)
prob = ODEProblem(opruga,u0,tspan,p)
sol = solve(prob)

plot(sol,vars=(1))
plot(sol,vars=(2))

function dva_tijela(du,u,p,t)
du[1] = u[2];
du[2] = (1/p[1])*(-p[3]*u[1]-p[4]*(u[2]-u[4]));
du[3] = u[4];
du[4]= (p[4]*(u[2]-u[4]))*(1/p[2])
end
\end{lstlisting}

%Ovo je \LaTeX\ predložak za izradu završnog rada prvog ciklusa studija, izrađen za potrebe studenata Elektrotehničkog fakulteta u Sarajevu. U radu se isprepliću dvije odvojene cjeline - sadržaj i forma. Sadržaj rada je baziran na idejama iz važećeg Pravilnika o strukturi i sadržaju doktorske disertacije i magistarskog rada na Elektrotehničkom fakultetu u Sarajevu (br. 04-1-673/11, dana 17.01.2011. godine), dok je forma rada definirana strukturom i načinom korištenja .tex fajlova.
%U fajlu Zavrsni\_rad\_BSc\_Ime\_Prezime.tex oblikovane su osnovne stranice, a naredbom \textit{include} umeću se dodatne stranice i dodaju poglavlja. 

%Osim \textit{Zavrsni\_rad\_BSc\_Ime\_Prezime.tex} fajla, koristi se još i: \textit{abstract.tex} (izdvojen fajl za sažetak na bosanskom i engleskom jeziku), \textit{postavka.tex} (izdvojen fajl za postavku rada), \textit{izjava.tex} (izdvojen fajl za izjavu o autentičnosti rada), \textit{poglavlje\_1.tex} (primjer jednog poglavlja),\textit{ prilog\_1.tex} (primjer jednog priloga) i \textit{literatura.bib} (bibliografski podaci).

%U sažetku je potrebno dati koncizan opis riješenih problema, metoda korištenih za njegovo (njihovo) rješavanje, dobivenih rezultata i zaključke. U sažetku se ne navode reference. Potrebno je voditi računa da se u sažetku ne daje uvod u rad, niti pregled poglavlja rada, već daje opis namjene rada do najviše 500 riječi.

\vspace{1cm}
\textbf{Ključne riječi}:  predložak, \LaTeX, ETF

\section*{Abstract}

This is a \LaTeX\ template for writing a bachelor's thesis, created for students of the Faculty of Electrical Engineering in Sarajevo. The work intertwines two separate parts - content and form. The content of the work is based on ideas from the current Regulations on the structure and content of doctoral dissertations and master's theses at the Faculty of Electrical Engineering in Sarajevo (no. 04-1-673/11, dated 17.01.2011), while the form of the work is defined by the structure and method of using .tex files.
In the file Zavrsni\_rad\_BSc\_Ime\_Prezime.tex, the basic pages are formatted, and additional pages and chapters are inserted using the \textit{include} command.

In addition to the \textit{Zavrsni\_rad\_BSc\_Ime\_Prezime.tex} file, the following are also used: \textit{abstract.tex} (separate file for summary in Bosnian and English), \textit{postavka.tex} (separate file for assignment), \textit{izjava.tex} (separate file for authenticity statement), \textit{poglavlje\_1.tex} (example of one chapter), \textit{prilog\_1.tex} (example of one appendix) and \textit{literatura.bib} (bibliographic data).

The abstract should provide a concise description of the solved problems, methods used to solve them, obtained results and conclusions. References are not cited in the abstract. Care should be taken that the abstract does not provide an introduction to the work, nor an overview of the chapters of the work, but rather gives a description of the purpose of the work up to a maximum of 500 words.

\vspace{1cm}
\textbf{Keywords}:  template, \LaTeX, ETF

\pagebreak

\section*{Sažetak}


\begin{lstlisting}[language=Julia]
using Pkg;
Pkg.add("DifferentialEquations")
using DifferentialEquations
function opruga(du,u,p,t)
    du[1] = u[2];
    du[2] = -p[1]*u[1]/p[2]
end

p = [100.0;10.0]
u0 = [0.0;1.0]
tspan = (0.0,10.0)
prob = ODEProblem(opruga,u0,tspan,p)
sol = solve(prob)

plot(sol,vars=(1))
plot(sol,vars=(2))

function dva_tijela(du,u,p,t)
du[1] = u[2];
du[2] = (1/p[1])*(-p[3]*u[1]-p[4]*(u[2]-u[4]));
du[3] = u[4];
du[4]= (p[4]*(u[2]-u[4]))*(1/p[2])
end
\end{lstlisting}

%Ovo je \LaTeX\ predložak za izradu završnog rada prvog ciklusa studija, izrađen za potrebe studenata Elektrotehničkog fakulteta u Sarajevu. U radu se isprepliću dvije odvojene cjeline - sadržaj i forma. Sadržaj rada je baziran na idejama iz važećeg Pravilnika o strukturi i sadržaju doktorske disertacije i magistarskog rada na Elektrotehničkom fakultetu u Sarajevu (br. 04-1-673/11, dana 17.01.2011. godine), dok je forma rada definirana strukturom i načinom korištenja .tex fajlova.
%U fajlu Zavrsni\_rad\_BSc\_Ime\_Prezime.tex oblikovane su osnovne stranice, a naredbom \textit{include} umeću se dodatne stranice i dodaju poglavlja. 

%Osim \textit{Zavrsni\_rad\_BSc\_Ime\_Prezime.tex} fajla, koristi se još i: \textit{abstract.tex} (izdvojen fajl za sažetak na bosanskom i engleskom jeziku), \textit{postavka.tex} (izdvojen fajl za postavku rada), \textit{izjava.tex} (izdvojen fajl za izjavu o autentičnosti rada), \textit{poglavlje\_1.tex} (primjer jednog poglavlja),\textit{ prilog\_1.tex} (primjer jednog priloga) i \textit{literatura.bib} (bibliografski podaci).

%U sažetku je potrebno dati koncizan opis riješenih problema, metoda korištenih za njegovo (njihovo) rješavanje, dobivenih rezultata i zaključke. U sažetku se ne navode reference. Potrebno je voditi računa da se u sažetku ne daje uvod u rad, niti pregled poglavlja rada, već daje opis namjene rada do najviše 500 riječi.

\vspace{1cm}
\textbf{Ključne riječi}:  predložak, \LaTeX, ETF

\section*{Abstract}

This is a \LaTeX\ template for writing a bachelor's thesis, created for students of the Faculty of Electrical Engineering in Sarajevo. The work intertwines two separate parts - content and form. The content of the work is based on ideas from the current Regulations on the structure and content of doctoral dissertations and master's theses at the Faculty of Electrical Engineering in Sarajevo (no. 04-1-673/11, dated 17.01.2011), while the form of the work is defined by the structure and method of using .tex files.
In the file Zavrsni\_rad\_BSc\_Ime\_Prezime.tex, the basic pages are formatted, and additional pages and chapters are inserted using the \textit{include} command.

In addition to the \textit{Zavrsni\_rad\_BSc\_Ime\_Prezime.tex} file, the following are also used: \textit{abstract.tex} (separate file for summary in Bosnian and English), \textit{postavka.tex} (separate file for assignment), \textit{izjava.tex} (separate file for authenticity statement), \textit{poglavlje\_1.tex} (example of one chapter), \textit{prilog\_1.tex} (example of one appendix) and \textit{literatura.bib} (bibliographic data).

The abstract should provide a concise description of the solved problems, methods used to solve them, obtained results and conclusions. References are not cited in the abstract. Care should be taken that the abstract does not provide an introduction to the work, nor an overview of the chapters of the work, but rather gives a description of the purpose of the work up to a maximum of 500 words.

\vspace{1cm}
\textbf{Keywords}:  template, \LaTeX, ETF

%%%%%%%%%%%%%%%%%%%%% SADRŽAJ %%%%%%%%%%%%%%%%%%%%%%%%%
%\clearpage
\tableofcontents

%%%%%%%%%%%%%%% POPIS SLIKA %%%%%%%%%%%%%%%%%%%%%%%%%%%
%\clearpage
\listoffigures
\addcontentsline{toc}{chapter}{Popis slika}

%%%%%%%%%%%%%%% POPIS TABELA %%%%%%%%%%%%%%%%%%%%%%%%%%
%\clearpage
\listoftables
\addcontentsline{toc}{chapter}{Popis tabela}

%%%%%%%%%%%%%%% POPIS OZNAKA %%%%%%%%%%%%%%%%%%%%%%%%%%%
\cleardoublepage % start new page
\pagestyle{fancyplain} % puts headers/footers back on
\fancyhf{}
\lhead{\nouppercase{\fancyplain{}{\leftmark}}}
\renewcommand{\chaptermark}[1]{\markboth{#1}{}}
\renewcommand{\footrulewidth}{0.4pt} %draw foot line
\lfoot{\slshape Mustafić A., Ljutika Z., "Nesterovljev ubrzani gradijentni metod za minimizaciju"}
\rfoot{\thepage}
\cfoot{}

%%%%%%%%%%%%%%%%%%%%%%%%%%%%%%%%%%%%%%%%%%%%%%%%%%%%%%%%%%%%%%%%%%%%
\mainmatter

%%%%%%%%%%%%%%%%%%%%%%%%%%%%%%%%%%%%%%%%%%%%%%%%%%%%%%%%%%%%%%%%%%%%
%%%%%%%%%%%%%%%%%%%%%%%%% POGLAVLJA %%%%%%%%%%%%%%%%%%%%%%%%%%%%%%%%
%%%%%%%%%%%%%%%%%%%%%%%%%%%%%%%%%%%%%%%%%%%%%%%%%%%%%%%%%%%%%%%%%%%%

%Poglavlja je najbolje raditi u odvojenim fajlovima
%Poglavlje 1
\chapter{Uvod}

%U skladu sa dobrom istraživačkom praksom, uvodno poglavlje rada prvog ciklusa studija bi trebao sadržavati bar sljedeće elemente:
%\begin{itemize}
%\item {obrazloženje teme,}
%\item {opis strukture rada.}
%\end{itemize}

%U narednom tekstu će detaljnije biti obrazložena svaka od tačaka.

\section{Obrazloženje teme}

Postupci minimizacije se mogu gledati kao jedno stablo. Na prvom stepenu možemo izabrati tip minimizacije i izabraćemo gradijentski. od svih mogućih gradijentskih, izabraćemo Nesterovljev ubrzani gradijentni metod.
Interesantan je, lići na nešto što bi radili na IF1. Ova tema je od izuzetnog značaja, jer nam ovisi ocjena o njoj, a neke od literatura koje smo koristili su nađene na internetu i knjigama.

%U ovoj sekciji autor je dužan da obrazloži koja tema ili problem će biti analizirani ili istraživani, te zbog čega je upravo ova tema pogodna i bitna za istraživanje. Pohvalno je napraviti pregled literature sa odgovarajućim referenciranjem na istu.

\section{Struktura rada}

Prvo ćemo dati teoretske osnove, zatim ćemo implementirati kod, vidjeti može li brže i vidjeti šta dalje.
%U ovoj sekciji je najbolje dati po jedan paragraf o svakom poglavlju iz rada. Potencirajte koji su glavni doprinosi svakog poglavlja, te kako su poglavlja medjusobno povezana.





%Poglavlje 2
\chapter{Implementacija algoritma}

Istraživanje se izlaže organizirano, koncizno i konzistentno kroz dva ili više odvojenih poglavlja, sa  pregledom teoretskih osnova na kojima su bazirana i kraćom diskusijom dobijenih rezultata. Ono što je izuzetno važno jeste da se dobiveni rezultati istraživanja konstantno objektivno porede sa postojećim rezultatima u literaturi ili oblasti istraživanja, te sistematično ukazuje na prednosti i nedostatke autorskog pristupa.
Izostanak komparacije rezultata istraživanja dobivenih od strane autora sa konkurentnim algoritmima, metodama i pristupima pokazuje nepostojanje akademske zrelosti, nedovoljnu posvećenost u istraživanju odgovarajuće naučne oblasti i vrlo često ukazuje na loš kvalitet disertacije/rada.

U radu se za formiranje poglavlja koriste sekcije, podsekcije, podpodsekcije, paragrafi i podparagrafi kao u primjerima koji slijedi.

\section{Primjer sekcije}
Ovo je primjer sekcije. Ovo je primjer sekcije. Ovo je primjer sekcije. Ovo je primjer sekcije. Ovo je primjer sekcije. Ovo je primjer sekcije. Ovo je primjer sekcije. Ovo je primjer sekcije. Ovo je primjer sekcije. Ovo je primjer sekcije. Ovo je primjer sekcije. Ovo je primjer sekcije. Ovo je primjer sekcije. Ovo je primjer sekcije. Ovo je primjer sekcije. Ovo je primjer sekcije. Ovo je primjer sekcije.
\subsection{Primjer podsekcije}
Ovo je primjer podsekcije. Ovo je primjer podsekcije. Ovo je primjer podsekcije. Ovo je primjer podsekcije. Ovo je primjer podsekcije. Ovo je primjer podsekcije. Ovo je primjer podsekcije. Ovo je primjer podsekcije. Ovo je primjer podsekcije. Ovo je primjer podsekcije. Ovo je primjer podsekcije. Ovo je primjer podsekcije. Ovo je primjer podsekcije. Ovo je primjer podsekcije. Ovo je primjer podsekcije.
\subsubsection{Primjer podpodsekcije}
Ovo je primjer podpodsekcije. Ovo je primjer podpodsekcije. Ovo je primjer podpodsekcije. Ovo je primjer podpodsekcije. Ovo je primjer podpodsekcije. Ovo je primjer podpodsekcije. Ovo je primjer podpodsekcije. Ovo je primjer podpodsekcije. Ovo je primjer podpodsekcije. Ovo je primjer podpodsekcije. Ovo je primjer podpodsekcije. Ovo je primjer podpodsekcije. Ovo je primjer podpodsekcije.
\paragraph{Primjer paragrafa}
Ovo je primjer paragrafa. Ovo je primjer paragrafa. Ovo je primjer paragrafa. Ovo je primjer paragrafa. Ovo je primjer paragrafa. Ovo je primjer paragrafa. Ovo je primjer paragrafa. Ovo je primjer paragrafa. Ovo je primjer paragrafa. Ovo je primjer paragrafa. Ovo je primjer paragrafa. Ovo je primjer paragrafa. Ovo je primjer paragrafa. Ovo je primjer paragrafa.
\subparagraph{Primjer podparagrafa}
Ovo je primjer podparagrafa. Ovo je primjer podparagrafa. Ovo je primjer podparagrafa. Ovo je primjer podparagrafa. Ovo je primjer podparagrafa. Ovo je primjer podparagrafa. Ovo je primjer podparagrafa. Ovo je primjer podparagrafa. Ovo je primjer podparagrafa. Ovo je primjer podparagrafa. Ovo je primjer podparagrafa. Ovo je primjer podparagrafa. Ovo je primjer podparagrafa. Ovo je primjer podparagrafa.

\begin{center}
    \vspace{0.5cm}
    *\hspace{2cm}*\hspace{2cm}*
    \vspace{0.3cm}
\end{center}

Na kraju svakog poglavlja, najbolje je dati jedan kraći zaključak koji ukratko objedinjuje sve najvažnije zaključke iz tog poglavlja. Taj kraći zaključak treba da služi kao poveznica između poglavlja koje se upravo završilo, i narednog poglavlja koje tek treba da počne. Ovaj zaključak je poželjno odvojiti bilo kao odvojenu podsekciju poglavlja nazvanu "Zaključak", bilo kao jednostavno izdvojeni dio teksta razmaknut zvjezdicama.

Kratak primjer zaključka za ovo poglavlje: U ovom poglavlju je pokazano kako se formiraju centralna poglavlja u radu/disertaciji. U nastavku će biti pokazano kako se piše konačan zaključak, te dati određeni tehnički podaci oko formatiranja teksta, slika i formula.


\begin{lstlisting}[language=Julia]
using Pkg;
Pkg.add("DifferentialEquations")
using DifferentialEquations
function opruga(du,u,p,t)
    du[1] = u[2];
    du[2] = -p[1]*u[1]/p[2]
end

p = [100.0;10.0]
u0 = [0.0;1.0]
tspan = (0.0,10.0)
prob = ODEProblem(opruga,u0,tspan,p)
sol = solve(prob)

plot(sol,vars=(1))
plot(sol,vars=(2))

function dva_tijela(du,u,p,t)
du[1] = u[2];
du[2] = (1/p[1])*(-p[3]*u[1]-p[4]*(u[2]-u[4]));
du[3] = u[4];
du[4]= (p[4]*(u[2]-u[4]))*(1/p[2])
end
\end{lstlisting}
%Poglavlje 3
\chapter{Primjene algoritma u praksi}

Istraživanje se izlaže organizirano, koncizno i konzistentno kroz dva ili više odvojenih poglavlja, sa  pregledom teoretskih osnova na kojima su bazirana i kraćom diskusijom dobijenih rezultata. Ono što je izuzetno važno jeste da se dobiveni rezultati istraživanja konstantno objektivno porede sa postojećim rezultatima u literaturi ili oblasti istraživanja, te sistematično ukazuje na prednosti i nedostatke autorskog pristupa.
Izostanak komparacije rezultata istraživanja dobivenih od strane autora sa konkurentnim algoritmima, metodama i pristupima pokazuje nepostojanje akademske zrelosti, nedovoljnu posvećenost u istraživanju odgovarajuće naučne oblasti i vrlo često ukazuje na loš kvalitet disertacije/rada.

U radu se za formiranje poglavlja koriste sekcije, podsekcije, podpodsekcije, paragrafi i podparagrafi kao u primjerima koji slijedi.

\section{Primjer sekcije}
Ovo je primjer sekcije. Ovo je primjer sekcije. Ovo je primjer sekcije. Ovo je primjer sekcije. Ovo je primjer sekcije. Ovo je primjer sekcije. Ovo je primjer sekcije. Ovo je primjer sekcije. Ovo je primjer sekcije. Ovo je primjer sekcije. Ovo je primjer sekcije. Ovo je primjer sekcije. Ovo je primjer sekcije. Ovo je primjer sekcije. Ovo je primjer sekcije. Ovo je primjer sekcije. Ovo je primjer sekcije.
\subsection{Primjer podsekcije}
Ovo je primjer podsekcije. Ovo je primjer podsekcije. Ovo je primjer podsekcije. Ovo je primjer podsekcije. Ovo je primjer podsekcije. Ovo je primjer podsekcije. Ovo je primjer podsekcije. Ovo je primjer podsekcije. Ovo je primjer podsekcije. Ovo je primjer podsekcije. Ovo je primjer podsekcije. Ovo je primjer podsekcije. Ovo je primjer podsekcije. Ovo je primjer podsekcije. Ovo je primjer podsekcije.
\subsubsection{Primjer podpodsekcije}
Ovo je primjer podpodsekcije. Ovo je primjer podpodsekcije. Ovo je primjer podpodsekcije. Ovo je primjer podpodsekcije. Ovo je primjer podpodsekcije. Ovo je primjer podpodsekcije. Ovo je primjer podpodsekcije. Ovo je primjer podpodsekcije. Ovo je primjer podpodsekcije. Ovo je primjer podpodsekcije. Ovo je primjer podpodsekcije. Ovo je primjer podpodsekcije. Ovo je primjer podpodsekcije.
\paragraph{Primjer paragrafa}
Ovo je primjer paragrafa. Ovo je primjer paragrafa. Ovo je primjer paragrafa. Ovo je primjer paragrafa. Ovo je primjer paragrafa. Ovo je primjer paragrafa. Ovo je primjer paragrafa. Ovo je primjer paragrafa. Ovo je primjer paragrafa. Ovo je primjer paragrafa. Ovo je primjer paragrafa. Ovo je primjer paragrafa. Ovo je primjer paragrafa. Ovo je primjer paragrafa.
\subparagraph{Primjer podparagrafa}
Ovo je primjer podparagrafa. Ovo je primjer podparagrafa. Ovo je primjer podparagrafa. Ovo je primjer podparagrafa. Ovo je primjer podparagrafa. Ovo je primjer podparagrafa. Ovo je primjer podparagrafa. Ovo je primjer podparagrafa. Ovo je primjer podparagrafa. Ovo je primjer podparagrafa. Ovo je primjer podparagrafa. Ovo je primjer podparagrafa. Ovo je primjer podparagrafa. Ovo je primjer podparagrafa.

\begin{center}
    \vspace{0.5cm}
    *\hspace{2cm}*\hspace{2cm}*
    \vspace{0.3cm}
\end{center}

Na kraju svakog poglavlja, najbolje je dati jedan kraći zaključak koji ukratko objedinjuje sve najvažnije zaključke iz tog poglavlja. Taj kraći zaključak treba da služi kao poveznica između poglavlja koje se upravo završilo, i narednog poglavlja koje tek treba da počne. Ovaj zaključak je poželjno odvojiti bilo kao odvojenu podsekciju poglavlja nazvanu "Zaključak", bilo kao jednostavno izdvojeni dio teksta razmaknut zvjezdicama.

Kratak primjer zaključka za ovo poglavlje: U ovom poglavlju je pokazano kako se formiraju centralna poglavlja u radu/disertaciji. U nastavku će biti pokazano kako se piše konačan zaključak, te dati određeni tehnički podaci oko formatiranja teksta, slika i formula.


%Poglavlje 4
\chapter{Zaključak i diskusija}

Cilj ovog rada bio je sistematično predstaviti Nesterovljevu metodu ubrzanog gradijenta, od teorijskih osnova, kroz implementaciju, do praktičnih primjena.
U nastavku ćemo rezimirati ključne zaključaka i diskutirati perfomans metode, relativan u odnosu na srodne izbore, te dotaći se otvorenih pitanja, mogućih pravaca daljnjeg istraživanja unutar domene teme.

% ─────────────────────────────────────────────────────────────────────────────
\section{Rezime pređenih tema}

Nesterovljeva metoda ubrzanog gradijentnog predstavlja jedan od teoretskih
najznačajnijih doprinosa u oblasti numeričke minimizacije i optimizacije. Nesterov je 1983. godine
dokazao da je stopa konvergencije $O(1/k^2)$ teoretski optimalna za metode prvog reda
na klasi glatkih konveksnih funkcija~\cite{nesterov1983method}, i da njegova metoda tu granicu
dostiže. Ovaj rezultat nije prevaziđen u narednih četrdeset godina.

U radu su implementirane i analizirane još dodatne dvije formulacije metode - Sutskeverova
i Bengiova, izvedene iz praktičnih potreba.
Eksperimentalni rezultati na Rosenbrockavoj funkciji, prikazani u
Tablici~\ref{tab:nag_results}, potvrđuju da su sve tri formulacije matematički ekvivalentne
u smislu pronalaska minimuma, ali se razlikuju u brzini konvergencije i numeričkim
karakteristikama. Sutskeverova formulacija pokazala je najbrže dostizanje kriterija
zaustavljanja, dok je Bengiova bila najstabilnija u smislu oscilacija oko minimuma.

Vremenom, praktična primjena metode dostigla je visoku rasprostranjenost u raznim oblastima numeričke minimizacije, gdje smo se u ovom radu fokusirali na dvije najveće direktne kontribucije metode:
dubokog učenja, gdje je rad Sutskevera i koautora~\cite{sutskever2013} označio
prekretnicu u treniranju dubokih i rekurentnih neuronskih mreža, te obrade signala, gdje
Nesterovljevo ubrzanje čini teorijsku osnovu FISTA algoritma~\cite{beck2009} i njegovih
primjena u CT rekonstrukciji i kompresiranom opažanju.

% ─────────────────────────────────────────────────────────────────────────────
\section{Diskusija performansi i ograničenja}

Ipak, pored opisanih snažnih odlika metode, NAG u praksi dolazi s nekoliko
značajnih ograničenja koja je potrebno razmotriti.

\subsection{Osjetljivost na hiperparametre}

NAG zahtijeva ručno podešavanje dvije ključne hiperparametre: stope učenja $\alpha$ i
koeficijenta momentuma $\mu$. Za razliku od adaptivnih metoda poput Adama,
koje automatski skaliraju korake učenja za svaki parametar pojedinačno i adaptiraju se
prema dinamici gradijenata tokom treniranja, NAG ne implementira nikakvo automatsko
prilagođavanje ovih vrijednosti.
To znači da korisnik mora pažljivo odabrati $\alpha$ i $\mu$ prije početka optimizacije, a taj izbor može značajno uticati na performanse algoritma.

U eksperimentima
prikazanim u ovom radu, fiksni parametri $\alpha = 0.001$ i $\mu = 0.9$ pokazali su se
zadovoljavajućim za Rosenbrockovu funkciju, međutim ovi parametri nisu
prenosivi. Problem sa drugačijom geometrijom funkcije gubitka generalno zahtijeva
ponovnu kalibraciju.

Loš odabir ovih parametara može rezultirati divergencijom ili
ekstremno sporom konvergencijom. Na ovo su Sutskever i saradnici strogo ukazivali i u svom inicijalnom osvrtu.

\subsection{Ograničenje na nekonveksnim funkcijama}

Nesterovljeve garancije konvergencije $O(1/k^2)$ vrijede isključivo za glatke konveksne
funkcije. Funkcije gubitka u modernom dubokom učenju \textbf{nisu konveksne, sadrže sedlaste
    tačke, ravne regije i lokalne minimume}.
U tim uvjetima, NAG nema formalnu garanciju
konvergencije, a ubrzanje u odnosu na standardni gradijentni spust nije teorijski
zagarantovano.
Empirijski uspjeh metode u dubokom učenju često je više zasnovan na čistim
heurističkim, a ne teoretskim osnovama~\cite{sutskever2013}.

\subsection{Poređenje sa Adamom}

U savremenom dubokom učenju, Adam optimizator~\cite{kingma2014} i njegovi izvodi
(AdamW, Nadam) dominiraju u praktičnoj upotrebi, a NAG se rjeđe koristi kao samostalan
optimizator.
Razlog leži u adaptivnoj prirodi Adama, automatskim podešavanjem efektivne
stope učenja po parametru, Adam je otporniji na loš odabir hiperparametara i brže
konvergira u ranim iteracijama. Međutim, istraživanja pokazuju da dobro podešen NAG
može postići komparabilne ili bolje rezultate od Adama, posebno gdje je
moguće alocirati veću računsku moć za podešavanje hiperparametara~\cite{sutskever2013}.

Izbor između NAG-a i Adam/Nadam optimizatora u praksi svodi se na sljedeće: NAG
je teorijski opravdan i razumni izbor za konveksne i dobro uvjetovane probleme,
dok Adam i Nadam dominiraju u dubokom učenju gdje je adaptivnost važnija od teoretskih
garancija.

% ─────────────────────────────────────────────────────────────────────────────
\section{Pravci daljnjeg istraživanja}

Na osnovu analize provedene u radu, mogu se identificirati nekoliko potencijalnih
pravaca za daljnje istraživanje i razvoj.

\textbf{Adaptivno raspoređivanje momentuma} (\textit{momentum scheduling}), mjesto
fiksnog $\mu$, dinamičko povećanje koeficijenta momentuma tokom iteracija moglo bi
poboljšati stabilnost konvergencije, naročito u ranim fazama optimizacije kada su
gradijenati zašumljeni i nepouzdani.

\textbf{Primjena FISTA-e u medicinskom snimanju}, kako u zdravstvenom sektoru, CT rekonstrukcija i MRI
kompresija postaju sve zahtjevniji s povećanjem rezolucije slike, primjena FISTA i varijanti algoritma
predstavljaju obećavajući pravac istraživanja u kojima brzina konvergencije bi znatno
uticalo na klinički tok rada.

\textbf{Veza sa kontinuiranim dinamičkim sistemima} — novija istraživanja pokazuju da
se NAG može interpretirati kao diskretizacija određene obične diferencijalne jednačine
drugog reda~\cite{su2016}, što otvara mogućnost analize stabilnosti i dizajna novih
optimizatora kroz teoriju dinamičkih sistema.

% ─────────────────────────────────────────────────────────────────────────────
\section{Zaključak}

Nesterovljeva metoda ubrzanog gradijentnog ostaje jedan od temelja moderne
numeričke minimizacije. Unatoč jednostavnosti koncepta nad kojim se oslanja,
dovoljna je da podstakne razvoj cijele porodice algoritama opšteprimjene, Sutskeverova i Bengiova formulacija omogućile su primjenu u dubokom učenju,
FISTA je prenijela Nesterovljevo ubrzanje u oblast obrade signala, a putem Nadam-a je čak pronašao svrhu i u najpopularnijem modernom optimizatoru.

Razumijevanje njene strukture i ograničenja predstavlja solidnu osnovu za praćenje i doprinos aktuelnim
istraživanjima u oblasti optimizacije i mašinskog učenja.
Činjenica da je metoda, objavljena 1980. godine, i danas prisutna
u vodećim okvirima za duboko učenje i ostvaruje nivo konvergencije koji još uvijek nije značajno prevažiđen, svjedoči o
njenoj fundamentalnoj važnosti kroz vremena.

Razumijevanje strukture metode, kao i ograničenja predstavlja solidnu osnovu za praćenje i doprinos aktuelnim
istraživanjima u oblasti numeričke minimizacije, mašinskog učenja, te čak i obrade signala.
U svakom od tih slučajeva, suštinska ideja ostaje ista, umjesto da reaguje na
gradijent tamo gdje se trenutno nalazi, metoda \textit{'osjeti'} budući položaj i ispravlja
kurs unaprijed. Ta jednostavna modifikacija, kako je pokazano u ovom radu, ima veoma duboke posljedice u eksponencijalnom razvoju algoritama optimizacije i njihovoj primjeni u savremenim problemima.


%%%%%%%%%%%%%%%%%%%%% PRILOZI %%%%%%%%%%%%%%%%%%%%%%%%%%%%%%%%%%%%
\begin{appendices}
  %Priloge je najbolje raditi u odvojenim fajlovima
  %Prilog 1
  \input{CHAPTERS/prilog_1}
\end{appendices}

%%%%%%%%%%%%%%%%%%%%%%%%%%%%%%%%%%%%%%%%%%%%%%%%%%%%%%%%%%%%%%%%%%%
\backmatter

%%%%%%%%%%%%%%%%%%%%%%% LITERATURA %%%%%%%%%%%%%%%%%%%%%%%%%%%%%%%%%
\addcontentsline{toc}{chapter}{Literatura}
\bibliographystyle{BIB/IEEEtranETF}
\bibliography{BIB/literatura}

%%%%%%%%%%%%%%%%%%% INDEKS POJMOVA %%%%%%%%%%%%%%%%%%%%%%%%%%%%%%%%
\addcontentsline{toc}{chapter}{Indeks pojmova}
\printindex


\end{document}
